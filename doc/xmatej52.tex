\documentclass[11pt,a4paper,titlepage]{article}
\usepackage[left=1.5cm,text={18cm, 25cm},top=2.5cm]{geometry}
\usepackage[utf8]{inputenc}
\usepackage{setspace}
\usepackage{graphicx}
\usepackage[czech]{babel}
\usepackage{float}
\usepackage{color}
\usepackage{hyperref}
\usepackage{fancyvrb}
\setlength{\parindent}{0cm}
\setlength{\parskip}{1em}
\sloppy

\hypersetup{
	colorlinks=true,
	linktoc=all,
	linkcolor=blue,
	citecolor=red,
	urlcolor=blue,
}

\begin{document}

		\setstretch{0.5}
		\begin{center}

			\includegraphics[width = 150mm]{logo.png}\\

			\vspace{\stretch{0.382}}

			\LARGE
			Přenos dat, počítačové sítě a protokoly\\
			Hromadný projekt - Hybridní chatovací P2P síť\\
			\vspace{\stretch{0.618}}

		\end{center}

	\Large{\today} \hfill Jiří Matějka (xmatej52)
	\thispagestyle{empty}
	\newpage
	\setcounter{page}{1}

    \tableofcontents
	\newpage
	\newpage

    \section{Úvod} \label{uvod}
        Tato práce vznikla jako projekt do předmětu Přenos dat, počítačové sítě a protokoly na škole
		Vysoké učení technické v~Brně. Práce se zabývá implementace chatovací sítě nad transportním protokolem
        UDP. Síť se skládá z libovolného počtu peerů a minimálně jednoho registračního uzlu.

	\section{Struktura sítě} \label{struktura}
		Jak již bylo zmíněno v úvodu\ref{uvod}, celá síť se skládá z několika peerů a
        a alespoň jednoho registračnního uzlu. K registračním uzlům se připojují jednotliví peeři (uživatelé)
        a právě od registračního uzlu získává každý uživatel informace o dalších uživatelých v síti.

        Uzel dále může navázat spojení s jiným uzlem a navzájem si tak vyměnit záznamy o
        registrovaných uživatelích a známých uzlech. Díky spojení mezi jednotlivými uzly
        může následně registrační uzel poskytnout informace i o uživatelích, které nejsou
        registrovaní přímo u něj, ale registrovali se k některýmu z jeho sousedů.

        \begin{figure}[htbp]
            \begin{center}
                \includegraphics[scale=0.5]{base.png}
                \caption{Ukázka struktury sítě}
                \label{base}
            \end{center}
        \end{figure}

		\subsection{Peer}
            Každý uživatel v síti je reprezentován peerem. Peer je v síti identifikován unikátním uživatelským
            jménem, ipv4 adresou a portem. V rámci operačního systému je identifikován pomocí unikátního ID.
            Každý uživatel je registrován k právě jednomu registračnímu uzlu, který
            mu poskytuje informace o ostatních uživatelých v síti.

        \subsection{Uzel}
            Uzel je v síti identifikován svou ipv4 adresou a portem. Každý z uzlů si vede databázi známých uzlů a peerů. Uzel poskytuje informace o
            uživatelých v síti pouze těm uživatelům, kteří jsou registrováni u něj. Uzly si
            navzájem mezi sebou pravidelně vyměňují databáze peerů i uzlů. Pokud uzel obdrží informaci
            o neznámém uzlu v síti, naváže s ním spojení.

    \section{Komunikace}
        Komunikace jednotlivých prvků v síti je implementována pomocí jednoduchého komunikačního protokolu. Každá zpráva přenášena
        zpráva mezi dvěma prvky sítě je přenášena skrz UDP a před přenosem samotné zprávy skrz UDP je obsach zprávy
        bencodován.

        \subsection{Komunikační prokol}
            V rámci komunikačního protokolu bylo implementováno 8 různých zpráv. Každá z těchto 8 zpráv má syntaxi
            JSON a všechny jejich atributy jsou povinné. Společnými atributy je potom typ zprávy, který identifikuje,
            o jakou zprávu se jedná a txid, který nese unikátní identifikátor zprávy.

            \subsubsection{ACK}
                Zpráva \texttt{ACK} slouží k potvrzení doručení zpráv typu \texttt{GETLIST}, \texttt{LIST}, \texttt{MESSAGE} a \texttt{DISCONNECT}.
                V komunikačním protokolu je implementována zejména proto, že UDP negarantuje doručení. Po odeslání výše uvedených zpráv
                se na zprávu \texttt{ACK} čeká maximálně po dobu 2 sekund, poté se předpokládá, že zpráva nebyla doručena.

                Zpráva má dva atributy -- txid a type, kde txid nese identifikátor zprávy, kterou zpráva \texttt{ACK} potvrzuje.
                Výsledná struktura zprávy tedy bude vypadat následovně: \verb+{"type":"ack", "txid":<ushort>}+

            \subsubsection{HELLO}
                Zpráva \texttt{HELLO} slouží k registraci peeru k uzlu. Pro udržení spojení mezi peerem a uzlem odesílá peer zprávu
                \texttt{HELLO} každých 10 sekund. Zpráva obsahuje kromě atributý type a txid další 3 atributy identifikující
                uživatele -- username, ipv4 a port. Pokud uzel neobdrží od uživatele po dobu 30 sekund žádnou zprávu \texttt{HELLO},
                uživatele odhlásí. Uživatele odhlásí i tehdy, pokud obdrží zprávu \texttt{HELLO} s nulovou ipv4 adresou a portem.

                Struktara zprávy \texttt{HELLO} vypadá následovně:
\begin{verbatim}
{
    "type":"hello",
    "txid":<ushort>,
    "username":"<string>",
    "ipv4":"<dotted_decimal_IP>",
    "port":<ushort>
}
\end{verbatim}

            \subsubsection{UPDATE}
                \texttt{UPDATE} slouží k navázání spojení mezi 2 uzly a výměny si informací o registrovaných peerech a známých uzlech. Zpráva
                \texttt{UPDATE} tedy v sobě nese seznam všech známých uzlů odesílatele včetně seznamu peerů, které jsou k jednotlivým uzlům
                registrovány. Zpráva update se odesílá každé 2 sekundy a pokud uzel neobdrží od některého ze svých sousedů zprávu \texttt{UPDATE}
                po dobu 10 sekund, přeruší komunikaci s tímto uzlem.

                Struktara zprávy \texttt{UPDATE} vypadá následovně:
\begin{verbatim}
{
    "type":"update",
    "txid":<ushort>,
    "db":{
        "<dotted_decimal_IP>,<ushort_port>":{
            "<ushort>":{
                "username":"<string>",
                "ipv4":"<dotted_decimal_IP>",
                "port":<ushort>
            }
        }
    }
}
\end{verbatim}

            \subsubsection{DISCONNECT}
                Po obdržení zprávy \texttt{DISCONNECT} přeruší uzel komunikaci s odesílatelem (za předpokladu, že odesílatel je registrační uzel).
                \texttt{DISCONNECT} obsahuje pouze atributy typ a txid a její struktara vypadá následovně: \verb+{"type":"disconnect", "txid":<ushort>}+
            \subsubsection*{GETLIST}
                Příkazu \texttt{GETLIST} využívá peer k aktualizaci svého seznamu známých uživatelů v síti. Zpráva \texttt{GETLIST} obsahuje pouze
                atributy txid a type a její strukturlze popsat následovně: \verb+{"type":"getlist", "txid":<ushort>}+
            \subsubsection{LIST}
                Zpráva \texttt{LIST} je odpovědí uzlu na příkaz \texttt{GETLIST} od peera. Zpráva nese v sobě seznam všech známých
                uživatelů v síti bez ohledu na to, k jakému uzlu jsou registrovaní.

                Zpráva \texttt{LIST} vypadá následovně:
\begin{verbatim}
{
    "type":"list",
    "txid":<ushort>,
    "peers":{
        "<ushort>":{
            "username":"<string>",
            "ipv4":"<dotted_decimal_IP>",
            "port":<ushort>
        }
    }
}
\end{verbatim}
            \subsubsection{MESSAGE}
                Zprávu \texttt{MESSAGE} použije peer, pokud chce zaslat zprávu jinému peeru. Posílání zpráv \texttt{MESSAGE}
                probíhá pouze mezi peery navzájem a registračních uzlů se tato zpráva vůbec netýká. \texttt{MESSAGE} kromě
                atributů txid a type nese ještě atribut identifikující odesílatele (from), příjemce (to) a atribut nesoucí obsah
                zprávy (message).
\begin{verbatim}
{
    "type":"message",
    "txid":<ushort>,
    "from":"<string>",
    "to":"<string>",
    "message":"<string>"
}
\end{verbatim}

            \subsubsection{ERROR}
                Tato zpráva se odesílá jako oznámení o chybě (neobdržení zprávy \texttt{ACK}, špatný formát přijaté zprávy, neregistrovaný užiatel
                odeslal uzlu příkaz \texttt{GETLIST} apod.). Kromě atributů txid a type se ve zprávě nachází jště atribut verbose, jehož
                hodnota nese textový popis chyby. Zpráva \texttt{ERROR} vypadá tedy následovně: \verb+{"type":"error", "txid":<ushort>, "verbose":"<string>"}+

    \section{Implementace}
        Projekt je implementován v jazyce python3 a samotná implementace je rozdělena do 7 modulů, které jsou společné pro uzly i peery.
        Veškerý příjem zpráv je implementován v modulu \texttt{Receiver.py}, odesílání zpráv v modulu \texttt{Sender.py}, registrace k uzlu,
        navázání a udržení spojení uzly a udržení spojení mezi uzlem a peerm v modulu \texttt{ConnectionKeeper.py}. Komunikační protokol je
        implementován v modulu \texttt{Protokol.py}. Zbylé 3 moduly jsou \texttt{Functions.py},
        kde jsou obsaženy užitečné funkce používané napříč mezi moduly (např. zpracování argumentů), \texttt{FileLock.py} sloužící k výlučnému přístupu
        k souborům a \texttt{InputReader.py} který načítá příkazy ze souborů a standartního vstupu.

        \subsection{Registrační uzel}
            Aplikace registračního uzlu je implementována v souboru \texttt{pds18-node.py}. Aplikaci lze ovládat zadáním příkazů na
            standartní vstup nebp pomocí aplikace RPC, která je popsána níže. Spuštění aplikace:
            \verb+pds18-node.py --id N --reg-ipv4 IPV4 --reg-port PORT+

            Po spuštění aplikace uzel vytvoří vlákno, které pravidelně kontroluje databázi známých sousedů a případně odesílá
            zprávy typu \texttt{UPDATE}. Dále vytvoří vlákno, které pravidelně kontroluje databázi registrovaných peerů a
            známých uzlů a případně tuto databázi promazává. Další vytvořené vlákno zpracovává přijaté zprávy a tvoří na ně
            odpovědi. A poslední 2 vlákna, co vzniknou načítají příkazy ze souboru (kam je zadává aplikace RPC) a ze standartního
            vstupu (kam je zadává sám uživatel). Rodičovský proces pouze tyto příkazy předává ostatním vláknům a případně vypisuje
            chyby nebo zpracované informace na chybový nebo standartní výstup.

            Podporované příkazy zadané na standartní vstup:
            \begin{itemize}
                \item \verb+\c ipv4 port+ -- Naváže spojení se zadaným uzlem
                \item \verb+\s+ -- Vynutí synchronizaci s ostatnímy uzly
                \item \verb+\l+ -- Vypíše aktualní databázi uzlů a jejich peerů
                \item \verb+\n+ -- Vypise databázi znamých uzlů
                \item \verb+\d+ -- Odpojí se od ostatnich uzlů
                \item \verb+\exit+ -- Ukončí aplikaci
            \end{itemize}

    \section{Testování}
    \section{Závěr}


\end{document}

\documentclass[11pt,a4paper,titlepage]{article}
\usepackage[left=1.5cm,text={18cm, 25cm},top=2.5cm]{geometry}
\usepackage{setspace}
\usepackage{graphicx}
\usepackage[czech]{babel}
\usepackage[latin2]{inputenc}
\usepackage{float}
\usepackage{color}
\usepackage{hyperref}
\setlength{\parindent}{0cm}
\setlength{\parskip}{1em}
\sloppy

\hypersetup{
	colorlinks=true,
	linktoc=all,
	linkcolor=blue,
	citecolor=red,
	urlcolor=blue,
}

\begin{document}

		\setstretch{0.5}
		\begin{center}

			\includegraphics[width = 150mm]{logo.png}\\

			\vspace{\stretch{0.382}}

			\LARGE
			Přenos dat, počítačové sítě a protokoly\\
			Hromadný projekt - Hybridní chatovací P2P síť\\
			\vspace{\stretch{0.618}}

		\end{center}

	\Large{\today} \hfill Jiří Matějka (xmatej52)
	\thispagestyle{empty}
	\newpage
	\setcounter{page}{1}

    \tableofcontents
	\newpage
	\newpage

    \section{Úvod} \label{uvod}
        Tato práce vznikla jako projekt do předmětu Přenos dat, počítačové sítě a protokoly na škole
		Vysoké učení technické v~Brně. Práce se zabývá implementace chatovací sítě nad transportním protokolem
        UDP. Síť se skládá z libovolného počtu peerů a minimálně jednoho registračního uzlu.

	\section{Struktura sítě} \label{struktura sítě}
		Jak již bylo zmíněno v úvodu\ref{uvod}, celá síť se skládá z několika počtu peerů a
        a alespoň jednoho registračnního uzlu. K registračním uzlům se připojují jednotliví peeři (uživatelé)
        a právě od registračního uzlu získává každý uživatel informace o dalších uživatelých v síti.

        Uzel dále může navázat spojení s jiným uzlem a navzájem si tak vyměnit záznamy o
        registrovaných uživatelích a známých uzlech. Díky spojení mezi jednotlivými uzly
        může následně registrační uzel poskytnout informace i o uživatelích, které nejsou
        registrovaní přímo u něj, ale registrovali se k některýmu z jeho sousedů.


		\subsection{Peer}


		\subsection{Poruchy nastávající během zpracování čokolády}
			Na všech strojích na lince může dojít k poruchám.

			\subsubsection{Mixér}
				Na mixéru nastávají hned 2 poruchy.
				První poruchou je ucpání filtru. To je způsobeno,
				když je čokoláda nedokonale vyjmuta z obalu a~kousek igelitového
				obalu je vložen do mixéru spolu s čokoládou. Filtr se časem ucpe,
				a~musí se vyměnit v~intervalu daném exponenciálním rozložením se
				středem 43 dnů. Doba opravy trvá 1~--~4 hodiny rovnoměrně.

				Druhá a~závažnější porucha nastává na čepeli mixéru. Tehdy se mixér musí rozebrat,
				čepel vyměnit a~stroj seřídit. Tato událost nastává s exponenciálním rozložením se
				středem 365 dnů. Doba opravy čepele a~seřízení stroje zabere 4 -- 7 dní rovnoměrně včetně víkendů.


			\subsubsection{Čerpadlo}
				Na čerpadle dochází pouze k jediné poruše, a~to je způsobeno zatvrdnutím
				čokolády. Čokoláda tvrdne po 15 minutách, kdy je v~nádobě.
				V případě poruchy se rozehřeje čerpadlo a~nádoba, a~následně do konce čerpání
				je ohřev puštěn. Doba než je čerpadlo možné opět spustit trvá 30 -- 40 minut rovnoměrně.

			\subsubsection{Stroj tvořící výrobky}
				Na tomto stroji dochází k zavzdušnění. Zavzdušnění stroje je dáno
				exponenciálním rozložením se středem 14 dní. Odvzdušnění je dáno
				exponenciálním rozložením se středem 15 minut. Tuto opravu zvládne zaměstnanec.

		\subsection{Obsluha zaměstnanci}
			Mixér, čerpadlo a~stroj na výrobu čokolády obsluhuje dohromady jen jeden
			zaměstnanec. Ten naplní mixér čokoládou (přibližně 3 -- 5 sekund na kilo) a~spustí ho.
			Během doby, co je mixér spuštěný, musí u něj zaměstnanec stát. Poté, co je čokoláda
			dostatečně tekutá, vypne mixér a~začne plnit nádobu. Jakmile čokoláda
			přeteče do nádoby, zaměstnanec jde spustit čerpadlo a~spolu s ním
			se spustí i stroj tvořící čokoládové výrobky. Zaměstnanec se posléze
			přemístí zpět k mixéru a~nádobě na čokoládu. v~nádobě je otvor, na
			který je napojeno čerpadlo, umístěn v~levé části, a~tam zaměstnanec
			hrne veškerou čokoládu. Poté co odčerpá poslední zbytky čokolády,
			vypíná se stroj na tvorbu výrobků, a~jde se opět plnit mixér.

			Detekce	zmetků je prováděna v~jiné části výroby. Toto je v~modelu
			zjednodušeno a~počítá se pouze s časovou prodlevou, než se vadné
			kusy vrátí zpět na začátek procesu. Model to vůbec neovlivní,
			protože při plnění mixéru je úplně jedno, zda vložený kus čokolády je
			zmetek nebo ještě nezpracovaný kus.

		\subsection{Upřesnění získaných dat} \label{upresneni_dat}
			Pro správné fungování simulačního modelu \cite[strany 44 a 49]{web:prednasky}
			bylo nutné upřesnit zadaná data. Z
			dalších konzultací s Pavlem Dubou jsme dospěli k výsledkům uvedených
			v tabulce \ref{table:data}. Přestože údaje jsou již přesnější, model
			systému bude třeba ještě validovat a~verifikovat \cite[strany 36 a 37]{web:prednasky}.

			\begin{table}[htb]
				\centering
			    \begin{tabular}{|l|l|l|}
				\hline
				    Údaj                            & Původní hodnota                      & Upravená hodnota            \\ \hline
					Doba vložení 1 kila do mixéru   & pár vteřin                           & 3 -- 5 sekund rovnoměrně    \\
				    Doba obsluhy mixéru             & 15 minut                             & 14 -- 17 minut rovnoměrně   \\
				    Doba čerpání 1 kila             & 12 sekund                            & 12 sekund                   \\
				    Doba zpracování 1 kila          & 18 sekund                            & 18 sekund                   \\ \hline
				    Porucha čepele na mixéru        & 1 krát do roka                       & (exp) 365 dnů               \\
				    Zanesení filtru na mixéru       & 1 krát max 2 krát za 2 měsíce        & (exp) 43 dnů                \\
				    Zatvrdnutí čokolády             & 15 minut 	                           & 15 minut                    \\
				    Zavzdušnění stroje              & 1 krát za 14 dní                     & (exp) 14 dní                \\ \hline
				    Doba opravy čepele mixéru       & Do týdne to je opravené              & 4 -- 7 dní rovnoměrně       \\
				    Výměna/vyčištění filtru         & 1 -- 4 hodiny, záleží na technikovi  & 1 -- 4 hodiny rovnoměně     \\
				    Oprava čerpadla                 & 30 -- 40 minut                       & 30 -- 40 minut rovnoměrně   \\
				    Odvzdušnění stroje              & kolem 15 minut                       & (exp) 15 minut              \\ \hline
			    \end{tabular}
				\caption{Tabulka zobrazující data získaná z průzkumu a~jejich upravené hodnoty na základě experimentu}
				\label{table:data}
			\end{table}

	\section{Rozbor tématu a použitých metod a technologií}
		Pro tvorbu simulačního modelu procesu zpracování čokolády je nezbytné
		znát, jak takový proces probíhá. Samotný proces je podrobně popsán v~kapitole
		\ref{zapracovani_cokolady}.

		\subsection{Popis použitých postupů}
			Pro tvorbu simulačního projektu byl použit jazyk C\texttt{++} s knihovnou
			SIMLIB \cite{web:Simlib}. C\texttt{++} je objektově orientovaný jazyk, a~spolu s knihovnou
			SIMLIB je ideálním nástrojem pro simulaci tohoto modelu.

		\subsection{Původ použitých metod a technologií}
			Autorem knihovny SIMLIB je Dr. Ing. Petr Peringer a~spoluautory Ing. David Leška a Ing. David Martinek. Použité konstrukce
			jsou převzaty z příkladů konstrukcí \cite{web:simlib_examples}.

	\section{Koncepce modelu}
		Cílem projektu je simulovat proces zpracování čokolády na lince a~ověřit
		efektivnost výroby, popřípadě navrhnout efektivnější systém. Simulace
		se nezaměřuje na celý proces zpracování, ale pouze na část, kde se čokoláda
		připravuje na další zpracování, tato část je společná pro výrobu většiny
		produktů firmy LION PRODUCTS, s.r.o. v~tomto modelu budeme pracovat s
		čokoládou typu Callebaut, pro kterou odpovídají hodnoty uvedené v
		tabulce \ref{table:data}, u ostatních druhů čokolády by se data mírně lišila.

		Nejprve jsme si vytvořili abstraktní model, reprezentovaný stochastickou Petriho sítí
		\cite[strany 123 -- 135]{web:prednasky},
		na základě informací zjištěných v~kapitole \ref{zapracovani_cokolady}.
		Tuto síť můžete vidět na obrázku \ref{fig:sit-komplet}.

		Kvůli její komlikovanosti jsme se ji však následně rozhodli rozdělit do několika obrázků,
		které by lépe vysvětlovaly jednotlivé situace, jako například poruchy na jednotlivých strojích.

		Obrázek \ref{fig:sit-pracovni_doba} reprezentuje pracovní dobu zaměstnance na lince.
		Poté, co zaměstnanec příjde do práce, se spustí 15 hodinový odpočet.
		Po jeho uplynutí zaměstnanec dokončí pracovní cyklus a~znovu již nenaplňuje mixér.
		Po 5 pracovních dnech následují 2 dny volna.

		Obrázek \ref{fig:sit-mixer} reprezentuje obě zminované poruchy mixéru z kapitoly \ref{zapracovani_cokolady}.
		Porucha je dána odpočtem, po jehož uplynutí dojde k narušení běžného chodu.
		Porucha čepele zablokuje mixér ihned a~stroj je nutné opravit. Veškerá čokoláda z mixéru je přesunuta do nádoby na zmetky.
		Porucha filtru nezastavuje běh mixéru, ale je opravována, jakmile zaměstnanec přemístí rozmixovanou čokoládu z mixéru do nádoby.

		Obrázek \ref{fig:sit-zavzdusneni} reprezentuje problém vznikající zavzdušněním stroje zpracovávajícího čokoládu.
		Jakmile zaměstnanec zjistí, že stroj je zavzdušněn, vyčká než dokončí stroj svoji činnost, a~následně ho odvzdušní.

		Obrázek \ref{fig:sit-tvrdnuti} reprezentuje tvrdnutí čokolády v~nádobě.
		Čokoláda po 15 minutách v~nádobě vychladne a~není možné ji čerpat.
		Po vychladnutí čokolády se musí zahřát nádoba s čokoládou, aby bylo čerpadlo schopné čokoládu znovu čerpat.
		To trvá 30 -- 40 minut rovnoměrně.

		Obrázek \ref{fig:sit-hlavni} reprezentuje výrobu čokolády bez jakýchkoliv poruch.
		Zaměstnanec naplní mixér, spustí mixování trvající 14 -- 17 minut.
		Rozmixovanou čokoládu přemístí do nádoby a~s tou se následně přemístí k čerpadlu.
		Čerpadlo a~stroj spustí, začne čokoládu míchat a~sunout k vývodu do čerpadla.
		Stroj mezitím vytváří čokoládové výrobky, které jsou následně kontrolovány, zda jsou kvalitní.
		Pokud se narazí na zmetky, jsou přemístěny do nádoby, která koresponduje s nádobou na zmetky,
		odkud nabírá čokoládu zaměstnanec.

		\begin{figure}
		\begin{center}
			\includegraphics[scale=0.3]{../navrh/detaily/pracovni_doba.png}
			\caption{Pracovní doba}
			\label{fig:sit-pracovni_doba}
		\end{center}
		\end{figure}

		\begin{figure}
		\begin{center}
			\includegraphics[scale=0.3]{../navrh/detaily/poruchy_mixeru.png}
			\caption{Poruchy mixéru}
			\label{fig:sit-mixer}
		\end{center}
		\end{figure}

		\begin{figure}
		\begin{center}
			\includegraphics[scale=0.3]{../navrh/detaily/zavzdusneni.png}
			\caption{Zavzdušnění stroje}
			\label{fig:sit-zavzdusneni}
		\end{center}
		\end{figure}

		\begin{figure}
		\begin{center}
			\includegraphics[scale=0.3]{../navrh/detaily/tvrdnuti.png}
			\caption{Tvrdnutí čokolády}
			\label{fig:sit-tvrdnuti}
		\end{center}
		\end{figure}

		\begin{figure}
			\begin{center}
				\includegraphics[scale=0.33]{../navrh/detaily/hlavni_cast.png}
				\caption{Hlavní část}
				\label{fig:sit-hlavni}
			\end{center}
		\end{figure}

		\begin{figure}
		\begin{center}
			\includegraphics[scale=0.25]{../navrh/navrh-petriho_sit-cerna.png}
			\caption{Kompletní síť}
			\label{fig:sit-komplet}
		\end{center}
		\end{figure}

	\newpage
	\section{Architektura simulačního modelu}
		V implementaci je Zaměstnanec reprezentován třídou Zamestnanec, čokoláda, která
		již opustila čerpadlo je reprezentována třídou Cokolada. Stroje na lince (mixér, čerpadlo a~stroj na zpracování čokolády)
		jsou reprezentovány pomocí třídy Facility \cite[strana 180]{web:prednasky}.
		Porucha zanesení filtru je implementována pomocí třídy Ucpani\_filtru, porucha na
		čepeli mixéru pomocí třídy Porucha\_cepele, zavzdušnění stroje pomocí třídy
		Zavzdusneni a~tvrdnutí čokolády pomocí třídy Tvrdnuti\_cokolady.

		Třídy Zamestnanec, Cokolada, Porucha\_cepele a~Ucpani\_filtru jsou potomky třídy Process \cite[strana 171]{web:prednasky}
		a předkem tříd Tvrdnuti\_cokolady a~Zavzdusneni je třída Event \cite[strana 169]{web:prednasky}

	\section{Podstata simulačních experimentů a jejich průběh}
		Pomocí simulačních experimentů \cite[strana 9]{web:prednasky} chceme zjistit, jaké je ideální množství
		čokolády vkládané do mixéru a~zjištění, jak moc velký dopad mají poruchy na produkci
		čokolády.

		\subsection{Simulace provozu linky}
			Tento experiment \cite[strana 9]{web:prednasky} má za cíl prokázat validitu simulačního modelu. Tento experiment
			proběhl několikrát a~vedl k upřesnění získaných dat. Na tento experiment jsme se
			odkazovali v~kapitole \ref{zapracovani_cokolady}, a~výsledkem tohoto experimentu
			je tabulka \ref{table:data} s upřesněnými statistickými údaji a~graf \ref{fig:graf_ctvrtleti}
			znázorňující výrobu čokolády a~společný vliv poruch na produktivitu linky.

			\begin{figure}[H]
			\begin{center}
				\includegraphics[scale=0.65]{../statistika/graf_ctvrtleti_50.png}
				\caption{ Výroba čokoládových výrobků za dobu 10 let }
				\label{fig:graf_ctvrtleti}
			\end{center}
			\end{figure}

		\subsection{Zjištění ideálního množství čokolády vkládané do mixéru}
			Z grafu \ref{fig:graf_ctvrtleti} lze mimo jiné vyčíst, že při aktuálním
			provozu linky nedochází k tvrdnutí čokolády. Pomocí tohoto experimentu
			chceme zjistit, v~jakém množství vkládané čokolády do mixéru začne
			docházet k tvrdnutí čokolády a~v jakém množství je produkce nejvyšší.

			Výsledkem této práce je graf \ref{fig:graf_tvrdnuti}, ze kterého lze zjistit,
			že linka má nejlepší výsledky při vkládní 75 kilogramů čokolády do mixéru.
			Z tabulky \ref{table:experiment_1} lze zase zjistit, že
			při tomto množství je linka schopna vyrobit až o 22.2 \% více výrobků.

			\begin{table}[htb]
				\centering
				\begin{tabular}{|l|l|l|l|l|l|l|}
				\hline
				Množství čokolády [kg]  & Zatvrdnutí čokolády & Vyrobeno čokolady [t] \\ \hline
				50                 & 0                   & 3677.936 \\
				55                 & 0                   & 3802.846 \\
				60                 & 0                   & 4008.138 \\
				65                 & 0                   & 4217.805 \\
				70                 & 0                   & 4342.173 \\
				74                 & 0                   & 4406.111 \\
				75                 & 0                   & 4495.423 \\
				76                 & 2558                & 4408.304 \\
				77                 & 4773                & 4186.009 \\
				78                 & 6971                & 4064.702 \\
				79                 & 8893                & 3967.276 \\
				80                 & 10823               & 3906.272 \\
				85                 & 18271               & 3517.709 \\
				90                 & 23745               & 3185.453 \\
				95                 & 27931               & 2984.075 \\
				100                & 30941               & 2785.517 \\ \hline
				\end{tabular}
				\caption{ Data získaná ze simulace }
				\label{table:experiment_1}
			\end{table}

			\begin{figure}[H]
			\begin{center}
				\includegraphics[scale=0.65]{../statistika/graf_vztah_vstupu_a_tvrdnuti.png}
				\caption{ Vztah vkládaného množství čokolády do mixéru a~výsledného množství výrobků }
				\label{fig:graf_tvrdnuti}
			\end{center}
			\end{figure}

		\subsection{Vliv jednotlivých poruch na krátkodobou produkci čokolády}
			Výsledek toho experimentu lze použít pro zjištění dopadu jednotlivých poruch
			na produkci a~lze jeho výsledky použít například k analýze, zda se nevyplatí
			některý stroj nahradit za jiný.

			\subsection{Vliv zavzdušnění stroje}
				Stroj na zpracování čokolády má nejvíce poruch ze všech ostatních strojů na lince.
				Pokud výrazně ovlivňuje produkci čokolády, bylo by vhodné hledat jiné řešení,
				jak zpracovávat čokoládu.

				Z výsledného grafu \ref{fig:graf_stroj} a~tabulky \ref{table:experiment_stroj} lze
				zjistit, že stroj má minimální dopad na produkci. Produkce byla následující
				hodinu sice poloviční, ale na výslednou hodnotu produkce celého dne to má malý
				dopad a~vzhledem k tomu, že porucha nastává v~průměru jednou za 14 dní,
				dopad na produkčnost linky je vskutku minimální.
				\begin{figure}[H]
				\begin{center}
					\includegraphics[scale=0.65]{../statistika/graf_vliv_stroje.png}
					\caption{ Průběh výroby a~nastávání poruchy stroje během jednoho dne výroby }
					\label{fig:graf_stroj}
				\end{center}
				\end{figure}


				\begin{table}[H]
				    \centering
				    \begin{tabular}{|l|l|l|l|l|l|l|}
				    	\hline
					    Čas [h] &  Poruch stroje &  Čokoládových výrobků [t] \\ \hline
					    1 &  0 &  0.078 \\
					    2 &  0 &  0.1 \\
					    3 &  0 &  0.089 \\
					    4 &  0 &  0.093 \\
					    5 &  0 &  0.088 \\
					    6 &  1 &  0.091 \\
					    7 &  0 &  0.045 \\
					    8 &  0 &  0.089 \\
					    9 &  0 &  0.106 \\
					    10 &  0 &  0.095 \\
					    11 &  0 &  0.096 \\
					    12 &  0 &  0.093 \\
					    13 &  0 &  0.098 \\
					    14 &  0 &  0.09 \\
					    15 &  0 &  0.091 \\
					    16 &  0 &  0.043 \\
					    17 &  0 &  0 \\
					    18 &  0 &  0 \\
					    19 &  0 &  0 \\
					    20 &  0 &  0 \\
					    21 &  0 &  0 \\
					    22 &  0 &  0 \\
					    23 &  0 &  0 \\
					    24 &  0 &  0 \\ \hline
				    \end{tabular}
				    \caption{Vliv poruchy stroje na počet výrobků}
				    \label{table:experiment_stroj}
				\end{table}

			\subsection{Vliv zanesení filtru}
				Filtr se sice nezanáší tak často, ale jeho oprava může trvat přibližně 4 krát -- 16 krát
				déle. Výsledky této simulace jsou zobrazeny v~grafu \ref{fig:graf_filtr} a
				tabulce \ref{table:experiment_filtr}

				Přestože oprava filtru netrvala ani 2 hodiny, i tak dokázala snížit produkci čokolády
				v~následujících dvou hodinách o něco více než 50 \%. Ale vzhledem k tomu, že
				porucha nastává přibližně jednou za 43 dní, je to opět zanedbatelné.

				\begin{figure}[H]
				\begin{center}
					\includegraphics[scale=0.65]{../statistika/graf_vliv_filtru.png}
					\caption{ Průběh výroby a~nastání poruchy filtru během jednoho dne výroby }
					\label{fig:graf_filtr}
				\end{center}
				\end{figure}

				\begin{table}[H]
				    \centering
				    \begin{tabular}{|l|l|l|}
					    \hline
					    Čas [h] &  Poruch filtru &  Čokoládových výrobků [t] \\ \hline
					    1 &  0 &  0.078 \\
					    2 &  0 &  0.1 \\
					    3 &  0 &  0.089 \\
					    4 &  1 &  0.093 \\
					    5 &  0 &  0.045 \\
					    6 &  0 &  0.038 \\
					    7 &  0 &  0.096 \\
					    8 &  0 &  0.087 \\
					    9 &  0 &  0.092 \\
					    10 &  0 &  0.093 \\
					    11 &  0 &  0.088 \\
					    12 &  0 &  0.093 \\
					    13 &  0 &  0.09 \\
					    14 &  0 &  0.095 \\
					    15 &  0 &  0.095 \\
					    16 &  0 &  0.029 \\
					    17 &  0 &  0 \\
					    18 &  0 &  0 \\
					    19 &  0 &  0 \\
					    20 &  0 &  0 \\
					    21 &  0 &  0 \\
					    22 &  0 &  0 \\
					    23 &  0 &  0 \\
					    24 &  0 &  0 \\ \hline
				\end{tabular}
				\caption{Vliv poruchy filtru na počet výrobků}
				\label{table:experiment_filtr}
			\end{table}
		\subsection{Vliv poruch čepele}
			Porucha čepele má bezesporu největší vliv na produkci čokolády, otázkou
			je, zda ten dopad je tak významný, aby to mělo smysl řešit, a~zda je tento
			problém vůbec řešitelný.

			Graf \ref{fig:graf_cepel} znázorňuje průběh výroby a~z tabulky \ref{table:experiment_cepel}
			lze vyčíst, že porucha ovlivnila výrobu čokolády celkem v~6~dnech (z toho 2 byly víkendové dny).
			V prvním dnu snížila produkci o necelou třetinu, v~následujících 4 dnech byla produkce nulová
			a~6. den byla produkce minimální. Vliv poruchy čepele na produkci je dokonce viděl
			i~v~dlouhodobé produkci znázorněné na grafu \ref{fig:graf_ctvrtleti}.

			\begin{figure}[H]
			\begin{center}
				\includegraphics[scale=0.65]{../statistika/graf_vliv_filtru.png}
				\caption{ Průběh výroby a~nastání poruchy čepele během jednoho měsíce výroby }
				\label{fig:graf_cepel}
			\end{center}
			\end{figure}

			\begin{table}[htb]
			    \centering
			    \begin{tabular}{|l|l|l|}
			    \hline
			    Čas [den] & Poruch čepele &  Čokoládových výrobků [t] \\ \hline
			    1  & 0 & 1.434 \\
			    2  & 0 & 1.442 \\
			    3  & 0 &  1.46 \\
			    4  & 0 &  1.44 \\
			    5  & 0 &  1.427 \\
			    6  & 0 &  0 \\
			    7  & 0 &  0 \\
			    8  & 0 &  1.433 \\
			    9  & 0 &  1.447 \\
			    10 & 1 &  1.09 \\
			    11 & 0 &  0 \\
			    12 & 0 &  0 \\
			    13 & 0 &  0 \\
			    14 & 0 &  0 \\
			    15 & 0 &  0.174 \\
			    16 & 0 &  1.379 \\
			    17 & 0 &  1.464 \\
			    18 & 0 &  1.449 \\
			    19 & 0 &  1.412 \\
			    20 & 0 &  0 \\
			    21 & 0 &  0 \\
			    22 & 0 &  1.45 \\
			    23 & 0 &  1.434 \\
			    24 & 0 &  1.447 \\
			    25 & 0 &  1.406 \\
			    26 & 0 &  1.442 \\
			    27 & 0 &  0 \\
			    28 & 0 &  0 \\
			    29 & 0 &  1.443 \\
			    30 & 0 &  1.447 \\ \hline
			    \end{tabular}
			    \caption{Vliv poruchy čepele na počet výrobků}
			    \label{table:experiment_cepel}
			\end{table}
	\section{Závěr}
		Podle provedených experimentůch lze zvýšit produktivitu linky, a~to až o 22 \%.
		Toho lze docílit zvýšením množství čokolády vkládané do mixéru o 25 kilogramů (na 75 kg).
		Protože některé ze statistických údajů nebyly natolik přesné, a~protože simulační model
		nikdy nemůže být naprosto shodný s realitou, tak se mohou výsledky těchto experimentů
		provedených v~realitě trochu lišit. I přes tento fakt jsme došli k~závěru, že
		by bylo pro firmu LION PRODUCTS vhodné alespoň zkusit navýšit toto množství. Co se
		poruch týče, tak jediná porucha, která stojí za pozornost, je porucha čepele. U této
		poruchy se nám nepodařilo zjistit data, jak změnit poruchovost stroje a~otázka,
		jak řešit tento problém, je tedy nezodpovězena.
	\newpage
	\bibliography{zdroje}

\end{document}
